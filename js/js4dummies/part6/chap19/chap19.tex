% Created 2019-04-17 Mer 14:05
% Intended LaTeX compiler: pdflatex
\documentclass[presentation]{beamer}
\usepackage[utf8]{inputenc}
\usepackage[T1]{fontenc}
\usepackage{graphicx}
\usepackage{grffile}
\usepackage{longtable}
\usepackage{wrapfig}
\usepackage{rotating}
\usepackage[normalem]{ulem}
\usepackage{amsmath}
\usepackage{textcomp}
\usepackage{amssymb}
\usepackage{capt-of}
\usepackage{hyperref}
\usepackage[french, frenchb]{babel}
\hypersetup{colorlinks = true}
\usetheme{default}
\author{Laurent Garnier}
\date{}
\title{Dix bibliothèques et frameworks pour aller plus loin avec JavaScript}
\hypersetup{
 pdfauthor={Laurent Garnier},
 pdftitle={Dix bibliothèques et frameworks pour aller plus loin avec JavaScript},
 pdfkeywords={},
 pdfsubject={},
 pdfcreator={Emacs 26.1 (Org mode 9.1.14)}, 
 pdflang={Frenchb}}
\begin{document}

\maketitle
\begin{frame}{Outline}
\tableofcontents
\end{frame}


\section{Angular JS}
\label{sec:org8211a96}

\begin{frame}[label={sec:orga8d9219}]{Ressources pour Angular JS}
\begin{itemize}
\item \url{https://angularjs.org/}
\item \url{https://docs.angularjs.org/tutorial/}
\end{itemize}
\end{frame}

\begin{frame}[label={sec:orgbd7e6d3}]{Brève description d'Angular}
Angular JS a pour but :

\begin{itemize}
\item D'améliorer la testabilité du code en séparant manipulation du DOM
et logique de l'application
\item De tester plus facilement le code
\item De créer une séparation entre le côté client de l'application et
le côté serveur
\item De fournir une structure pour le processus de construction de
l'application, de la conception initiale à l'interface
utilisateur, en passant par la logique de l'écriture et le test du
code
\end{itemize}
\end{frame}

\begin{frame}[label={sec:org1c08fcf}]{Qui l'utilise ?}
\begin{itemize}
\item \href{https://www.youtube.com/channel/UC0-\_4-vctZIzlTRWnIqGfOA?view\_as=subscriber}{YouTube}
\item \url{https://www.lynda.com/}
\item \url{https://www.netflix.com/}
\item \url{https://www.freelancer.com/}
\end{itemize}
\end{frame}

\section{Backbone.js}
\label{sec:org9ca3b9a}

\begin{frame}[label={sec:org1f62aae}]{Ressources pour Backbone.js}
\begin{itemize}
\item \url{https://backbonejs.org/}
\item \url{https://en.wikipedia.org/wiki/Backbone.js}
\end{itemize}
\end{frame}


\begin{frame}[label={sec:orge261c16}]{Brève description de Backbone.js}
Backbone.js est une bibliothèque JavaScript MVC open source conçue
pour créer des sites Web avec une page unique. Le développement
d'applications Web avec Backbone donne une structure à celles-ci, et
renforce le très bon principe selon lequel la communication avec le
serveur devrait être effectuée \emph{via} une API RESTful.

MVC est un sigle qui signifie \emph{Model-View-Controller}. Son principe
consiste à séparer le modèle de données (\emph{Model}) de l'interface
utilisateur (\emph{View}) en utilisant une couche de contrôle
intermédiaire (\emph{Controller}). Cette couche appelle le modèle et la
vue voulus en fonction des données présentes en entrée. De son côté,
REST signifie \emph{Representational State Transfer} (transfert d'état
représentationnel). Il s'agit en bref d'un style d'architecture pour
le développement d'applications réseau s'appuyant sur un protocole
de communications client-serveur sans état. Les applications RESTful
utilisent typiquement des requêtes HTTP pour poster des données, les
lire ou encore les supprimer.
\end{frame}


\begin{frame}[label={sec:org236fbff}]{Qui l'utilise ?}
\begin{itemize}
\item \url{https://www.reddit.com/}
\item \url{https://bitbucket.org/}
\item \url{https://www.tumblr.com/}
\item \url{https://www.pinterest.fr/}
\item \url{https://www.linkedin.com/}
\end{itemize}
\end{frame}

\section{Ember.js}
\label{sec:orga214561}

\begin{frame}[label={sec:orgebfc068}]{Ressources pour Ember.js}
\begin{itemize}
\item \url{https://emberjs.com/}
\item \url{https://en.wikipedia.org/wiki/Ember.js}
\item \url{https://youtu.be/N4KrBuO0RRE}
\end{itemize}
\end{frame}


\begin{frame}[label={sec:org94c5e95}]{Brève description d'Ember}
Ember.js est l'un des plus anciens frameworks JavaScript MVC, ses
racines remontant à 2007. Il se définit lui-même comme << un
framework pour créer des applications Web ambitieuses >>. Comme la
plupart des autres frameworks décrit dans ce chapitre, il est basé
sur une architecture logicielle MVC. Et comme Backbone, il est conçu
pour créer des applications Web monopages.

Ember a la réputation d'être difficile à apprendre. Cependant, une
fois que vous le connaissez, ses bénéfices sont multiples. Avec du
code écrit selon les pratiques normales d'Ember, le développeur n'a
plus besoin de tout spécifier manuellement pour son application, ce
qui peut faire gagner beaucoup de temps.
\end{frame}

\begin{frame}[label={sec:orgbed8848}]{Qui l'utilise ?}
\begin{itemize}
\item \url{https://www.digitalocean.com/}
\item \url{https://vine.co/}
\item \url{http://nbcnews.com/}
\item \url{https://www.twitch.tv/}
\item \url{https://www.mediabistro.com/}
\end{itemize}
\end{frame}


\section{Famo.us}
\label{sec:orgf8248b1}

\begin{frame}[label={sec:org7fd73e9}]{Ressources pour Famo.us}
\begin{itemize}
\item \url{https://famous.co/}
\item \url{https://www.youtube.com/channel/UCiFhuK7AExmfhl8iUzw2g4w}
\item \url{https://github.com/famous}
\item \href{https://fr.slideshare.net/hinablue/famous-new-generation-of-html5-web-application-framework}{slideshare}
\end{itemize}
\end{frame}

\begin{frame}[label={sec:org587e4a5}]{Brève description de Famo.us}
Famo.us est un framework JavaScript open source servant à créer des
interfaces utilisateur complexes pour n'importe quel écran.

Famo.us contient un moteur de rendu 3D, ce qui rend possible
d'écrire du code JavaScript capable de déplacer des objets en 3D
dans le navigateur, et de créer des effets et des interfaces qui
n'étaient auparavant disponibles que dans des logiciels
spécialisés. De ce fait, les applications créées avec Famo.us
peuvent être plus rapides et plus fluides qu'avec une stricte
utilisation de HTML 5, CSS 3 et JavaScript.
\end{frame}

\begin{frame}[label={sec:org39ecba3}]{Qui l'utilise ?}
\begin{itemize}
\item \url{http://superstereo.co.uk/}
\item \url{https://japantoday.com/}
\end{itemize}
\end{frame}

\section{Knockout}
\label{sec:orga6f960f}

\begin{frame}[label={sec:org454b129}]{Ressources pour Knockout}
\begin{itemize}
\item \url{https://knockoutjs.com/}
\item \url{https://en.wikipedia.org/wiki/Knockout\_(web\_framework)}
\item \url{https://youtu.be/z3-G9bYNbwU}
\item \href{https://www.youtube.com/watch?v=yC9Lt6hTIUE\&list=PLo80fWiInSIONI-Al0iVvq9NNWllM0RrT}{Knockout tutorial (Indian)}
\end{itemize}
\end{frame}


\begin{frame}[label={sec:org10241da}]{Brève description de Knockout}
Knockout est un framework JavaScript open source servant à
simplifier la programmation d'interfaces utilisateur dynamiques. Il
utilise un modèle dit MVVM (\emph{Model-View-View-Model}) qui est un
dérivé du modèle MVC.

Parmi les caractéristiques de Knockout, mentionnons celles-ci : 
\begin{itemize}
\item liaisons déclaratives
\item rafraîchissement automatique de l'interface utilisateur lorsque
les données changent
\item suivi des dépendances
\item modélisation
\end{itemize}
\end{frame}

\begin{frame}[label={sec:org82347f9}]{Qui l'utilise ?}
\begin{itemize}
\item \url{https://www.mlb.com/}
\item \url{https://www.ancestry.com/}
\item \url{https://www.eventbrite.com/}
\item \url{https://www.tdameritrade.com/home.page}
\end{itemize}
\end{frame}

\section{QUnit}
\label{sec:org56a00d4}
\begin{frame}[label={sec:org0e04aed}]{Ressources pour QUnit}
\begin{itemize}
\item \url{https://qunitjs.com/}
\item \url{https://qunitjs.com/cookbook/}
\item \url{https://en.wikipedia.org/wiki/QUnit}
\item \url{https://github.com/qunitjs/qunit}
\end{itemize}
\end{frame}

\begin{frame}[label={sec:orge59b5b0}]{Brève description de QUnit}
QUnit est un framework destiné à la réalisation de tests pour
JavaScript. Il est utilisé pour de nombreux projets JavaScript open
source, y compris jQuery. Il peut tester n'importe quel code
JavaScript générique, et il est connu pour être aussi puissant que
facile à utiliser.
\end{frame}

\begin{frame}[label={sec:org195793a}]{Qui l'utilise ?}
\begin{itemize}
\item \url{https://jquery.com/}
\item \url{https://jqueryui.com/}
\item \url{https://jquerymobile.com/}
\item \url{https://www.sitepoint.com/}
\end{itemize}
\end{frame}


\section{Underscore.js}
\label{sec:orgcfdfbb6}
\begin{frame}[label={sec:orgf98c28a}]{Ressources pour Underscore.js}
\begin{itemize}
\item \url{https://underscorejs.org/}
\item \href{http://royto.familleroy.fr/2013/10/08/presentation-de-underscore-js/}{roytoblog}
\end{itemize}
\end{frame}

\begin{frame}[fragile,label={sec:org81f8650}]{Brève description de Underscore}
 Underscore est une bibliothèque JavaScript qui fournit aux
programmeurs de nombreuses fonctions utiles. Une fois que vous aurez
commencé à utiliser les fonctionnalités d'Underscore, vous vous
demanderez comment vous aviez pu vivre sans elles.

Parmi ces fonctionnalités offertes par Underscore, citons \texttt{sortBy}
(pour trier des listes), \texttt{groupBy} (pour regrouper une collection en
jeux d'éléments), \texttt{contains} (retourne \texttt{true} si une liste contient
une valeur spécifiée), \texttt{shuffle} (retourne une copie mélangée d'une
liste) et environ une centaine d'autres fonctions (dont la plupart
auraient dû être implémentées dans JavaScript dès l'origine).
\end{frame}

\begin{frame}[label={sec:org3717dd8}]{Qui l'utilise ?}
\begin{itemize}
\item \url{https://www.dropbox.com/}
\item \url{https://lifehacker.com/}
\item \url{https://www.theverge.com/}
\item \url{https://www.att.com/}
\item \url{https://gawker.com/}
\end{itemize}
\end{frame}


\section{Modernizr}
\label{sec:org57a1590}
\begin{frame}[label={sec:orgce5ea31}]{Ressources pour Modernizr}
\begin{itemize}
\item \url{https://modernizr.com/}
\item \url{https://fr.wikipedia.org/wiki/Modernizr}
\end{itemize}
\end{frame}

\begin{frame}[label={sec:orgea07932}]{Brève description de Modernizr}
Modernizr est une bibliothèque JavaScript permettant de détecter les
fonctionnalités du navigateur Web dans lequel il est exécuté.

Il est le plus souvent utilisé comme moyen très simple et pratique
pour vérifier si le navigateur de l'utilisateur est capable
d'exécuter un code JavaScript particulier, ou s'il faut utiliser
d'abord une API avant d'essayer d'exécuter ce code. Modernizr est
fréquemment employé en conjonction avec des outils appelés
\emph{Polyfills}, qui fournissent une méthode alternative pour mettre en
oeuvre certaines fonctionnalités avancées des navigateurs modernes
sur des dispositifs ou dans des navigateurs moins évolués.
\end{frame}

\begin{frame}[label={sec:orge1a67aa}]{Qui l'utilise ?}
\begin{itemize}
\item \url{http://go.com/}
\item \url{https://www.dotdash.com/}
\item \url{https://www.hostgator.com/}
\item \url{https://www.addthis.com/}
\item \url{https://eu.usatoday.com/}
\end{itemize}
\end{frame}

\section{Handlebars.js}
\label{sec:orge98088b}
\begin{frame}[label={sec:orgce8a4a1}]{Ressources pour Handlebars.js}
\begin{itemize}
\item \url{https://handlebarsjs.com/}
\item \url{https://fr.wikipedia.org/wiki/Handlebars\_(moteur\_de\_template)}
\end{itemize}
\end{frame}

\begin{frame}[label={sec:org8b15622}]{Brève description de Handlebars}
Handlebars est un moteur servant à créer des gabarits (ou
\emph{templates}) JavaScript côté client. Il permet d'insérer ces
gabarits dans des pages Web qui seront analysées afin d'utiliser les
données en temps réel qui sont passées à Handlebars.js
\end{frame}

\begin{frame}[label={sec:orgf55bab2}]{Qui l'utilise ?}
\begin{itemize}
\item \url{https://www.meetup.com}
\item \url{https://mashable.com/}
\item \url{https://www.flickr.com/}
\item \url{https://www.wired.com/}
\item \url{https://www.overstock.com/}
\end{itemize}
\end{frame}

\section{jQuery}
\label{sec:org34228e1}
\begin{frame}[label={sec:orge2dc62a}]{Ressources pour jQuery}
\begin{itemize}
\item \url{https://jquery.com/}
\item \url{https://fr.wikipedia.org/wiki/JQuery}
\end{itemize}
\end{frame}

\begin{frame}[label={sec:orgccfac94}]{Brève description de jQuery}
jQuery est la bibliothèque << Ecrire moins, en faire plus >> de
JavaScript. jQuery est utilisé par plus de 60\% des sites les plus
populaires du Web. Il est devenu un outil indispensable pour la
plupart des programmeurs JavaScript.
\end{frame}

\begin{frame}[label={sec:org15138d2}]{Qui l'utilise ?}
\begin{itemize}
\item \url{https://fr.wordpress.com/}
\item \url{https://www.pinterest.fr/}
\item \url{https://www.amazon.fr/}
\item \url{https://www.microsoft.com/fr-fr/}
\item \url{https://www.etsy.com/}
\end{itemize}
\end{frame}
\end{document}